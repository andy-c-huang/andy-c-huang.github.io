%\documentclass[12pt]{article}
\documentclass{article}

\renewcommand{\baselinestretch}{1.3}

\usepackage{tikz, amsmath, amsthm, amssymb, mathrsfs, fullpage, setspace}
%\usetikzlibrary{matrix,arrows}

\newcommand{\N}[0]{\mathbb{N}} \newcommand{\Z}[0]{\mathbb{Z}} \newcommand{\C}[0]{\mathbb{C}}
\newcommand{\R}[0]{\mathbb{R}} \newcommand{\F}[0]{\mathbb{F}} \newcommand{\Q}[0]{\mathbb{Q}}
\newcommand{\PP}[0]{\mathbb{P}} \newcommand{\AP}[0]{\mathbb{A}} \newcommand{\G}[0]{\mathbb{G}}

\newcommand{\ideal}[0]{\triangleleft}
\newcommand{\primeideal}[0]{\triangleleft _{prime}}
\newcommand{\maxideal}[0]{\triangleleft _{max}}
\newcommand{\m}[0]{\mathfrak m}
\newcommand{\p}[0]{\mathfrak p}
\newcommand{\tensor}[0]{\otimes}

\newcommand{\quotient}[2]{{\raisebox{.2em}{$#1$}\left/\raisebox{-.2em}{$#2$}\right.}}

\newcommand{\proven}[0]{\begin{flushright} \qed \end{flushright}}
\newcommand{\problem}[2]{\begin{flushleft} {\bf #1)} {\it #2} \end{flushleft} \hline \vline}
\newcommand{\dzbar}[0]{d\overline{z}}
\newcommand{\christoffel}[3]{\Gamma _{#1 #2} ^{#3}}

\begin{document}

\framebox{{\em Theorem:} Conformal metrics $\rho ^2 dz \odot d\overline{z}$ experience $K_\rho (z) = - \frac{\Delta log(\rho(z))}{\rho^2 (z)}$ Gauss curvature.}\newline



{\em Proof:} Given a two dimensional Riemannian manifold with conformal metric $(\Sigma, e^\phi (dx ^1 \tensor dx ^1 + dx^2 \tensor dx^2))$, we have $K = - \frac{R_{1212}}{(e^\phi)^2}$ by Gauss' Theorema Egregium. Furthermore, we know that $\christoffel{i}{j}{k}= \frac{1}{2} g^{kl} \left( \frac{\partial g_{il}}{\partial x^j} + \frac{\partial g_{jl}}{\partial x^i} + \frac{\partial g_{ij}}{\partial x^l} \right)$. Now, observe that at least two of the indices of the Christoffel symbol must agree, yielding:

\begin{eqnarray*}
(i=j=k) & \Rightarrow & \christoffel{i}{i}{i} = \frac{1}{2}g^{ii} \left( \frac{\partial g_{ii}}{\partial x^i} \right) = \frac{1}{2} e^{-\phi} \cdot \frac{\partial}{\partial x^i} \left( e^\phi \right) = \frac{1}{2}  \frac{\partial}{\partial x^i} \left(\phi \right) =\frac{1}{2} \phi _i, \\
(i=j)\neq k & \Rightarrow & \christoffel{i}{i}{k} = \frac{1}{2}g^{kk} \left( -\frac{\partial g_{ii}}{\partial x^k} \right) = \frac{1}{2} e^{-\phi} \cdot \frac{\partial}{\partial x^k} \left( e^\phi \right) = \frac{1}{2}  \frac{\partial}{\partial x^k} \left(\phi \right) = - \frac{1}{2} \phi _k, \\
(j=k)\neq i & \Rightarrow & \christoffel{i}{j}{j} = \frac{1}{2}g^{jj} \left( \frac{\partial g_{jj}}{\partial x^i} \right) = \frac{1}{2} e^{-\phi} \cdot \frac{\partial}{\partial x^i} \left( e^\phi \right) = \frac{1}{2}  \frac{\partial}{\partial x^i} \left(\phi \right) =\frac{1}{2} \phi _i, \\
(k=i)\neq j & \Rightarrow & \christoffel{i}{j}{i} = \frac{1}{2}g^{ii} \left( \frac{\partial g_{ii}}{\partial x^j} \right) = \frac{1}{2} e^{-\phi} \cdot \frac{\partial}{\partial x^j} \left( e^\phi \right) = \frac{1}{2}  \frac{\partial}{\partial x^j} \left(\phi \right) =\frac{1}{2} \phi _j. \\
\end{eqnarray*}
With this in hand, we compute $R_{1212}$ by definition (acquiring a formula along the way):

\begin{eqnarray*}
\left\langle R(\partial_1,\partial_2) \partial_1, \partial_2\right\rangle  & = & \left\langle \nabla _{\partial_1} \nabla _{\partial_2} \partial_1 - \nabla _{\partial_2} \nabla _{\partial_1} \partial_1 - \nabla _{[\partial_1, \partial_2]} \partial_1 , \partial_2\right\rangle \\
  & = & \left\langle \nabla _{\partial_1} \left( \christoffel{2}{1}{1} \partial_1 + \christoffel {2}{1}{2} \partial_2 \right) - \nabla _{\partial_2} \left( \christoffel{1}{1}{1} \partial_1 + \christoffel {1}{1}{2} \partial_2 \right), \partial_2\right\rangle \\
  & = & \left\langle \frac{\partial}{\partial x^1} \christoffel{2}{1}{1}\partial_1 +
\christoffel{2}{1}{1}\left( \christoffel{1}{1}{1}\partial_1 + \christoffel{1}{1}{2}\partial_2 \right) + \frac{\partial}{\partial x^1} \christoffel{2}{1}{2}\partial_2 + \christoffel{2}{1}{2}\left( \christoffel{1}{2}{1}\partial_1 + \christoffel{1}{2}{2}\partial_2 \right), \partial_2\right\rangle  \\
 & & \left\langle - \left[ \frac{\partial }{\partial x^2} \christoffel{1}{1}{1}\partial_1 + \christoffel{1}{1}{1}\left( \christoffel{2}{1}{1}\partial_1 + \christoffel{2}{1}{2}\partial_2 \right) +
\frac{\partial}{\partial x^2} \christoffel{1}{1}{2}\partial_2 + \christoffel{1}{1}{2}\left( \christoffel{2}{2}{1}\partial_1 + \christoffel{2}{2}{2}\partial_2 \right) \right] ,\partial_2 \right\rangle \\
 & = & \left[ \christoffel{2}{1}{1}\christoffel{1}{1}{2} + \frac{\partial}{\partial x^1} \christoffel{2}{1}{2} +
\christoffel{2}{1}{2}\christoffel{1}{2}{2} - \christoffel{1}{1}{1}\christoffel{2}{1}{2} -
\frac{\partial}{\partial x^2} \christoffel{1}{1}{2} - \christoffel{1}{1}{2}\christoffel{2}{2}{2} \right] \cdot \left\langle \partial_2 , \partial_2 \right\rangle \\
 & = & e^\phi \cdot \left[ \phi_2 \cdot ( - \phi_2 ) + 2 \frac{\partial}{\partial x^1}(\phi _1) + \phi _1 \cdot \phi _1 - \phi _1 \cdot \phi _1 -2\frac{\partial}{\partial x^2}(- \phi _2) - (-\phi_2)\cdot \phi _2 \right]\\
 & = & e^\phi \cdot \frac{1}{2} \Delta \phi \\
\end{eqnarray*}

Therefore, in the metric $e^\phi (dx ^1 \tensor dx ^1 + dx^2 \tensor dx^2)$, the Gauss curvature $K$ is given by
$$K = -\frac{e^\phi \Delta \phi}{2\cdot (e^\phi)^2} = - \frac{1}{2} e^{-\phi} \cdot \Delta \phi = -\frac{\Delta \phi}{2 e^\phi}.$$
Since $\rho ^2 dz \odot \dzbar$ was the form of our metric on $\Sigma$, $log(\rho ^2(z))$ plays the role of $\phi$ in the above. Hence, $$K_\rho (z) = -\frac{\Delta log (\rho ^2 (z))}{2 e^{log(\rho^2(z))}} = -\frac{\Delta log (\rho (z))}{\rho ^2 (z)}$$
\proven
\end{document}

​
