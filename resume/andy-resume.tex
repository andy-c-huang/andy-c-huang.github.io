\documentclass[margin,line]{res}
\usepackage{mathpazo, verbatim, hyperref}

%\setlength{\paperwidth}{8.5in}
%\setlength{\paperheight}{11in}

\topmargin=-0.3in
\oddsidemargin -.5in
\evensidemargin -.5in
\textwidth=6.2in
\textheight=9.5in
%\itemsep=0in
%\parsep=0in
% if using pdflatex:
\setlength{\pdfpagewidth}{8.5in}%\paperwidth}
\setlength{\pdfpageheight}{11in}%\paperheight}

\newenvironment{list1}{
  \begin{list}{\ding{113}}{%
      \setlength{\itemsep}{0in}
      \setlength{\parsep}{0in} \setlength{\parskip}{0in}
      \setlength{\topsep}{0in} \setlength{\partopsep}{0in}
      \setlength{\leftmargin}{0.17in}}}{\end{list}}
\newenvironment{list2}{
  \begin{list}{$\bullet$}{%
      \setlength{\itemsep}{0in}
      \setlength{\parsep}{0in} \setlength{\parskip}{0in}
      \setlength{\topsep}{0in} \setlength{\partopsep}{0in}
      \setlength{\leftmargin}{0.2in}}}{\end{list}}


\begin{document}
\moveleft.5\hoffset\centerline{\large\bf Andy C. Huang}
%\moveleft\hoffset\vbox{\hrule width 7.5in height 1pt}\smallskip
\moveleft.5\hoffset\centerline{{\it e-mail:}  andy.c.huang@outlook.com}
%\moveleft.5\hoffset\centerline{{\it webpage:} \url{http://math.rice.edu/~ach3}}
\moveleft.5\hoffset\centerline{{\it GitHub:} \url{https://github.com/andy-c-huang}}
\moveleft.5\hoffset\centerline{{\it cellphone:}  (408) 887 - 9651}
\vspace*{.1in}

\begin{resume}

%\section{\sc Objective} I am seeking a position in research and development allowing me to use the rigorous analytical skills I've developed from my doctoral training in mathematics. I thrive in an environment requiring collaboration with engineers from other areas,

\vspace*{-6pt}

\section{\sc Education and Honors}
{\bf Rice University}, Houston, TX \hfill
%\vspace*{-.1in}
\begin{list1}
\item[] Ph.D. in Mathematics (Advisor: Dr. Mike Wolf, GPA 3.96) \hfill May 2016
\item[] \hskip .2in {\em Dissertation Title:} "Handle crushing harmonic maps between surfaces"
\vskip -6pt
\item[] M.A. in Mathematics (GPA: 4.04) \hfill May 2012
\end{list1}
\vskip -12pt
{\bf University of California, Davis}, Davis, CA
\begin{list1}
\item[] B.S. in Mathematics with Departmental Citation (Major GPA: 3.85) \hfill May 2010
\end{list1}
\vskip -12pt
{\bf Budapest Semesters in Mathematics}, Budapest, Hungary \hfill Dec 2009

\section{\sc Computer skills}
{\em Programming languages:} C++, C, Python, bash scripting \\
{\em Software:} Git, MATLAB, Mathematica, CUBIT, ParaView, \LaTeX, tensorflow, scikit-learn, Trilinos\\
{\em Operating Systems:} Red Hat, Ubuntu, Windows

\section{\sc Professional Experience}
{\bf Sandia National Laboratories}, Albuquerque, NM \hfill {\bf Jan 2019 - present}\\
{\em Senior Member of the Technical Staff, Computer Science}
\vskip -5pt
Principal Investigator of a 3-year research project to utilize topological data analysis techniques to support machine learning development of equivalent circuit compact models of semiconductor devices. Team members include electrical engineers, applied mathematicians, and collaborations in academia.
\vskip -5pt
Code developer for two MPI-parallel and GPU supported C++ physics simulation codes, for semiconductor physics and electromagnetcs. Main work and research focus on numerical formulations, solver acceleration and stability, and physics implementation. On-going work include:
\begin{itemize}
\setlength\itemsep{0pt}
\item[-] Acceleration of frequency-domain analysis in semiconductor physics simulation code, \newline Charon (project homepage: \url{https://charon.sandia.gov}).
\item[-] Multi-physics coupling of some of Sandia's simulation codes, including Xyce.
\item[-] Electromagnetics simulation code solver acceleration for high-performance computing platforms.
\end{itemize}

{\bf Sandia National Laboratories}, Albuquerque, NM \hfill {\bf Jul 2016 - Jan 2019}\\
{\em Postdoctoral appointment (within Sandia's semiconductor device modelling team)}\\
Designed and implemented (in C++) a frequency-domain analysis method in Sandia's semiconductor device physics simulation code, Charon. Required extensive communication and cooperation with other developers to minimize disruption while extending code capability.

{\bf Sandia National Laboratories}, Albuquerque, NM \hfill {\bf May 2015 - Jun 2016}\\
{\em Graduate student intern, mentor: Xujiao (Suzey) Gao}\\
Compared and analyzed numerical stabilization schemes for finite element and control volume methods for the simulation of coupled transient drift-diffusion equations modelling semiconductor physics. Implemented these stabilization schemes in Sandia's TCAD program, Charon (in C++).

\section{\sc Previous Research and Projects}

{\bf Modeling of harmonic maps between surfaces \hfill Mar 2012 - Jun 2016}\\
Implementation of finite element methods in Matlab, and porting to Python with C libraries for optimization. Analyzed a transient semi-linear elliptic PDE. Obtained for examples considered in thesis.
\vskip -5pt
{\bf Developed a Windows 8.1 phone app: ``KnowYoNotes" \hfill Feb 01, 2015}\\
\emph{GitHub repo avilable at: }\url{https://github.com/andy-c-huang/KnowYoNotes}\\
Developed the interface for a sheet music reading learning app. Written in C\#, involving event driven programming. Further development ongoing. Created for Rice's 2015 24 hour hackathon, HackRice.
\vskip -5pt
{\bf Travelling wave phenomena: Wave reflection in excitable media \hfill Sept 2008 - Jun 2009} \newline
%{\em (in collaboration w/ Prof. Timothy Lewis (UC Davis), funded by the McNair Scholars program and the NSF)} \newline
Investigated wave reflection phenomena in modified, coupled Fitzhugh-Nagumo equations. Explored existence and stability of anti-phase behavior in periodic solutions - interpreted biologically as a cause for cardiac arrhythmia. McNair Scholars Program funded. Advisor: Dr. Timothy Lewis (UC Davis).
\vskip -5pt
{\bf Spread of Avian Influenza Across Heterogeneous Farmscapes \hfill Sept 2007 - Sept 2008} \\
%{\em (as an undergraduate cohort member of the 2007, UC Davis and NSF funded interdisciplinary Collaborative Learning at the Interface of Math and Biology group)} \\
Modeled spread of the H5N1 virus on farm networks of varying spatial heterogeneity using coupled, stochastic compartment models. Analyzed effect of heterogeneity on potential for a large outbreak. Presented to panel of poultry industry leaders, including Foster's Farms. UC Davis and NSF funded.

\section{\sc Grants and Awards}
\begin{enumerate}

\item Laboratory Directed Research and Development (LDRD) project award \hfill {\bf Sept 2019 - ongoing}\\
at Sandia National Laboratories, to lead a 3-year, 4-person team
\item SIAM Science Policy Fellowship \hfill {\bf Jan 2019 - Dec 2020}
\item GEAR Graduate Internship award \hfill {\bf Mar 2016} \\
to work with Dr. Domingo Toledo at the University of Utah in Salt Lake City, Utah
\item GEAR short-term visit award \hfill {\bf Jun 2015} \\
to work with Dr. Bill Goldman at the Mathematical Sciences Research Institute in Berkeley, CA
\item National Association of Math Circles Exchange grant \hfill {\bf 2014}
\item National Association of Math Circles Seed grant \hfill {\bf 2013}
\end{enumerate}

\section{\sc Publications and Presentations}
\begin{enumerate}
\item \emph{Physics-informed Modeling of Copper-doped Zinc Sulfide Elastomeric Composites} \hfill {\bf Sept 14, 2020}\\
      G. Hoover, J. Trujillo, B. Diehl, A. Huang, and D. Ryu\\
      ASME SMASIS Conference at Irvine, CA (in preparation)
\item \emph{Crack Prognosis using Mechanoluminescent Sensing Skins and Artificial Neural Networks} \hfill {\bf Apr 26, 2020}\\
      G. Hoover, J. Trujillo, S. Fakhimi, A. Huang, and D. Ryu\\
      SPIE NDE/SS Conference in Anaheim, CA (in preparation)
\item \emph{An Isofrequency Remapping Scheme for Harmonic Balance Methods} \hfill{\bf Feb 26, 2019}\\
      SIAM Conf. on Computational Science and Engineering in Spokane, WA
\item \emph{Analytic band-to-trap tunneling model including band offset for heterojunction devices} \hfill{\bf Feb 2019}\\
      Xujiao Gao, Bert Kerr, and Andy Huang\\
      In: Journal of Applied Physics 125, 054503 (2019)\\
      \url{https://doi.org/10.1063/1.5078685}
\item \emph{Non-degenerate harmonic balance via isofrequency remapping}\\
      (in preparation)
\item \emph{Developing a harmonic balance method for Charon}  \hfill {\bf Oct 24, 2018}\\
      Trilinos Users Group meeting in Albuquerque, NM
\item \emph{A versatile harmonic balance method in a parallel framework} \hfill {\bf Sept 25, 2018}\\
      A. Huang, X. Gao, R. Pawlowski, J. Gates, L. Musson, G. Hennigan, M. Negoita\\
      International Conference on Simulation of Semiconductor Processing and Devices (SISPAD)\\
      \url{http://dx.doi.org/10.1109/SISPAD.2018.8551620}
\item \emph{A harmonic balance method for PDEs} (SAND2016-12293)
\item \emph{Development of harmonic balance capability for Charon} (SAND2018-3623)
\item {\em Efficient band-to-trap tunneling model including heterojuntion band offset}\\
      Xujiao Gao, Andy Huang, and Bert Kerr\\
      In: ECS Transactions, vol. 80, pp. 1005–1015, 2017.
\item {\em Harmonic maps of punctured surfaces to the hyperbolic plane} \\%\hfill {\bf May 25, 2016} \\
      In: {\em Communications in Analysis and Geometry} (to appear) \\
      Preprint available at \url{http://arxiv.org/abs/1605.07715}
\item {\em Using geometric maximum principles to harmonically crush handles} \hfill {\bf Sep 10, 2016} \\
      GEAR Workshop: Analytic Aspects of Higher Teichm\"uller Theory\\
      at Rutgers University, Newark
\item {\em Handle crushing harmonic maps between surfaces} \hfill {\bf Mar 30, 2016} \\
      Max Dehn Seminar at the University of Utah, Math Department
\item {\em On harmonic maps between non-compact surfaces of different genera} \hfill {\bf Nov 15, 2015} \\
      AMS Special Session: Aspects of Minimal Surfaces in Riemannian Manifolds\\
      at Rutgers University
\item {\em Implementation of different stabilization schemes in Charon using Trilinos/Panzer} \hfill{ \bf Oct 27, 2015} \\
      X. Gao, A. Huang, K. Peterson, G. Hennigan, L. Musson, and P. Bochev \\
      Presented by X. Gao at the Trilinos User Group meeting Albuquerque, NM
\end{enumerate}

\section{\sc Professional development}
Enthought's Machine Learning Mastery Workshop (May 21-23 2018)\\
Kokkos tutorial (Dec. 5, 2017)\\
Enthought's Python for Scientists and Engineers (Sept. 11-15, 2017)\\
Introduction to CUBIT (Nov. 15, 2016)\\
ParaView (Aug. 3-5, 2016)

\section{\sc Conferences}
Intl. Conf. on Simulation of Semiconductor Processing and Devices in Udine, Italy (Sept. 4-6, 2019)\\
SIAM Conference on Computational Science and Engineering in Spokane, WA (Feb 25 - Mar 1, 2019)
Trilinos User-Developer Group meeting in Albuquerque, NM (Oct 23-25, 2018)\\
Intl. Conf. on Simulation of Semiconductor Processing and Devices in Austin, TX (Sept. 24-26, 2018)\\
IEEE Intl. Symp. on Antennas and Propagation APS/USRI in Boston, MA (Jul 8-13, 2018)\\
Trilinos User-Developer Group meeting in Albuquerque, NM (Oct. 23-25, 2017)\\
SIAM Conference on Computational Science and Engineering in Atlanta, GA (Feb 27 - Mar 3, 2017)

\section{\sc Mentorship and Teaching}
{\bf Head Mechanic at Rice Bikes \hfill Jul 2013 - May 2016} \\
Oversaw student-run bike shop mechanic training at Rice University. Organized bike builds for the rental bike fleet, using it as a training platform for team of 10. Coordinated monthly, themed bike rides for Rice students to safely tour areas of Houston. Re-structured mechanics operations for efficiency.
\vskip -5pt
{\bf Rice Math Circle co-organizer \hfill Aug 2012 - May 2014}\\
%Graduate student operated math circle for middle- and high-school aged students.
Coordinated and designed 3-hour, bi-weekly sessions run by graduate students exploring math topics through problems, puzzles, and games. Middle- and high-school student audience. Topics included: graph theory, discrete probability, combinatorics, game theory, minimal surfaces, programming.
\vskip -5pt
{\bf Instructor (at Rice University)} \\
Ordinary Differential Equations (Math 211), Calculus II (Math 102) \hfill {\bf Spring 2013, Summer 2012}
\vskip -5pt
{\bf Teaching Assistant \hfill Sept 2010 - May 2016}\\
Conducted office hours, led weekly discussions, and graded homework. {\em Courses:} Single and multi-variable calculus, Ordinary and partial differential equations, Riemannian geometry, Real Analysis

\section{\sc Service and Outreach}
{\bf Math Counts volunteer \hfill Feb 10, 2018}\\
scored exams from Math COUNTS competition at Albuquerque Academy
\vskip -5pt
{\bf Community Involvement brainstorm \hfill Dec 12, 2016}\\
Participated in a discussion lead by Tineca Quintana (org 3652)
\vskip -5pt
{\bf ``Changing the Equation" Career Spotlight interview \hfill Dec 6, 2016}\\
used in Explora math kits designed for after school activities, put together by Amy Tapia (org 3652)
\vskip -5pt
{\bf Mars Bus Experience volunteer \hfill Oct 27, 2016}\\
at Van Buren Middle School, organized by Tineca Quintana (org 3652)
\vskip -5pt
{\bf Invited presenter: ``How to be efficient (in some sense)" \hfill Mar 14, 2015} \\
as part of a Pi Day celebration at NASA's Space Center Houston
\vskip -5pt
{\bf Minimal surfaces and soap films demonstration \hfill Nov 4, 2012} \\
as part of the Weiss School of Natural Sciences alumni open house at Rice University

\begin{comment}
\section{\sc Hobbies and Interests}
Bicycling, rock climbing, racquetball, Arduino hacks
\end{comment}


\end{resume}
\end{document}




